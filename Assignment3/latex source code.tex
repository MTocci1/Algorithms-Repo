%%%%%%%%%%%%%%%%%%%%%%%%%%%%%%%%%%%%%%%%%
%
% CMPT xxx
% Some Semester
% Lab/Assignment/Project X
%
%%%%%%%%%%%%%%%%%%%%%%%%%%%%%%%%%%%%%%%%%

%%%%%%%%%%%%%%%%%%%%%%%%%%%%%%%%%%%%%%%%%
% Short Sectioned Assignment
% LaTeX Template
% Version 1.0 (5/5/12)
%
% This template has been downloaded from: http://www.LaTeXTemplates.com
% Original author: % Frits Wenneker (http://www.howtotex.com)
% License: CC BY-NC-SA 3.0 (http://creativecommons.org/licenses/by-nc-sa/3.0/)
% Modified by Alan G. Labouseur  - alan@labouseur.com
%
%%%%%%%%%%%%%%%%%%%%%%%%%%%%%%%%%%%%%%%%%

%----------------------------------------------------------------------------------------
%	PACKAGES AND OTHER DOCUMENT CONFIGURATIONS
%----------------------------------------------------------------------------------------

\documentclass[letterpaper, 10pt,DIV=13]{scrartcl} 

\usepackage[T1]{fontenc} % Use 8-bit encoding that has 256 glyphs
\usepackage[english]{babel} % English language/hyphenation
\usepackage{amsmath,amsfonts,amsthm,xfrac} % Math packages
\usepackage{sectsty} % Allows customizing section commands
\usepackage{graphicx}
\usepackage[lined,linesnumbered,commentsnumbered]{algorithm2e}
\usepackage{listings}
\usepackage{parskip}
\usepackage{lastpage}
\usepackage{lineno}
\usepackage{float}

\allsectionsfont{\normalfont\scshape} % Make all section titles in default font and small caps.

\usepackage{fancyhdr} % Custom headers and footers
\pagestyle{fancyplain} % Makes all pages in the document conform to the custom headers and footers

\fancyhead{} % No page header - if you want one, create it in the same way as the footers below
\fancyfoot[L]{} % Empty left footer
\fancyfoot[C]{} % Empty center footer
\fancyfoot[R]{page \thepage\ of \pageref{LastPage}} % Page numbering for right footer

\renewcommand{\headrulewidth}{0pt} % Remove header underlines
\renewcommand{\footrulewidth}{0pt} % Remove footer underlines
\setlength{\headheight}{13.6pt} % Customize the height of the header

\numberwithin{equation}{section} % Number equations within sections (i.e. 1.1, 1.2, 2.1, 2.2 instead of 1, 2, 3, 4)
\numberwithin{figure}{section} % Number figures within sections (i.e. 1.1, 1.2, 2.1, 2.2 instead of 1, 2, 3, 4)
\numberwithin{table}{section} % Number tables within sections (i.e. 1.1, 1.2, 2.1, 2.2 instead of 1, 2, 3, 4)

\setlength\parindent{0pt} % Removes all indentation from paragraphs.

\binoppenalty=3000
\relpenalty=3000

%----------------------------------------------------------------------------------------
%	TITLE SECTION
%----------------------------------------------------------------------------------------

\newcommand{\horrule}[1]{\rule{\linewidth}{#1}} % Create horizontal rule command with 1 argument of height

\title{	
   \normalfont \normalsize 
   \textsc{CMPT 435 - Fall 2023 - Dr. Labouseur} \\[10pt] % Header stuff.
   \horrule{0.5pt} \\[0.25cm] 	% Top horizontal rule
   \huge Assignment Three  \\     	    % Assignment title
   \horrule{0.5pt} \\[0.25cm] 	% Bottom horizontal rule
}

\author{Alex Tocci \\ \normalsize Mason.Tocci1@Marist.edu}

\date{\normalsize\today} 	% Today's date.

\begin{document}
\maketitle % Print the title

%----------------------------------------------------------------------------------------
%   Binary Search Tree
%----------------------------------------------------------------------------------------
\section{Binary Search Tree}
\subsection{Node Class}
% Insert C++ code with line numbers
\begin{linenumbers}
\begin{lstlisting}[language=C++, caption={Node Class}, label={code:example}]
#include <iostream>
#include <fstream>
#include <string>
#include <list>
// Include for std::random_device
#include <random>
#include <vector>
#include <sstream>
#include <iomanip>
#include <queue>

using namespace std;

// Define a Node class for the binary tree
class Node {
public:
    string data;
    Node* left;
    Node* right;

    Node(string item) : data(item), left(nullptr), right(nullptr) {}
};
\end{lstlisting}
\end{linenumbers}
\nolinenumbers

\textbf{Lines 1-10:} First we must include the necessary C++ libraries for our program to run. \\
\textbf{Line 12:} This allows us to use standard C++ functions without the prefix 'std::'. \\
\textbf{Lines 15-19:} Define a C++ class called 'Node' which contains three variables, 'string data', 'Node* left', and 'Node* right'. 'string data' stores the data associated with the node. 'Node* left' points to the left child node and 'Node* right' points to the right child node. \\
\textbf{Line 21:} This is a Node class constructor, which initializes the Node object.

\subsection{BST Insert Function}
% Insert C++ code with line numbers
\begin{linenumbers}
\begin{lstlisting}[language=C++, caption={BST Insert Function}, label={code:example}]
// Function to insert a node into the BST and print the path
Node* insert(Node* root, string item) {
    if (root == nullptr) {
        return new Node(item);
    }

    if (item < root->data) {
        cout << "L, ";
        root->left = insert(root->left, item);
    }
    else if (item > root->data) {
        cout << "R, ";
        root->right = insert(root->right, item);
    }

    return root;
}
\end{lstlisting}
\end{linenumbers}
\nolinenumbers

\textbf{Line 24:} BST insert function which initially takes in the root node of the BST and the string being inserted into the BST. \\
\textbf{Lines 25-27:}  If the current node 'root' is null pointer, return a new Node, passing in item as the data. \\
\textbf{Lines 29-32:}  If item is less than the data stored in the current node 'root', then print 'L, ' and recursively call the insert function with the left child of the current node. \\
\textbf{Lines 33-36:}  If item is greater than the data stored in the current node 'root', then print 'R, ' and recursively call the insert function with the right child of the current node. \\
\textbf{Line 38:} Return root, which is the current node.

\subsection{BST In Order Traversal Function}
% Insert C++ code with line numbers
\begin{linenumbers}
\begin{lstlisting}[language=C++, caption={BST In Order Traversal Function}, label={code:example}]
// In-order traversal function
void inOrderTraversal(Node* root) {
    if (root == nullptr) {
        return;
    }

    inOrderTraversal(root->left);
    cout << root->data << endl;
    inOrderTraversal(root->right);
}
\end{lstlisting}
\end{linenumbers}
\nolinenumbers

\textbf{Line 41:} BST inOrderTraversal function which initially takes in the root node of the BST. \\
\textbf{Lines 42-41:} If the current node 'root' is null pointer then return, as there are no nodes to visit. \\
\textbf{Line 46:} Recursively call the inOrderTraversal function passing in the left child of the current node 'root'. This visits all the nodes in the left subtree. \\
\textbf{Line 47:} Print the data in the current node 'root'. \\
\textbf{Line 48:} Recursively call the inOrderTraversal function passing in the right child of the current node 'root'. This visits all the nodes in the right subtree. \\

\subsection{BST Lookup Function}
% Insert C++ code with line numbers
\begin{linenumbers}
\begin{lstlisting}[language=C++, caption={BST Lookup Function}, label={code:example}]
// Function to look up an item in the BST and print the path
bool lookup(Node* root, string item, int& comparisons, string& path) {
    if (root == nullptr) {
        return false;
    }

    // Increment comparisons
    comparisons++;
    if (item == root->data) {
        return true;
    }
    else if (item < root->data) {
        path += "L, ";
        return lookup(root->left, item, comparisons, path);
    }
    else {
        path += "R, ";
        return lookup(root->right, item, comparisons, path);
    }
}
\end{lstlisting}
\end{linenumbers}
\nolinenumbers

\textbf{Line 51:} BST lookup function which initially takes in the root node of the BST, the string we are looking for, an int to count comparisons, and a string to keep track of the path. \\
\textbf{Lines 52-54:}  If the current node 'root' is null pointer then the item does not exist in the BST and return false.
\textbf{Line 57:} Increment 'comparisons' to keep track of the number of comparisons made. \\
\textbf{Lines 58-60:}  If the string 'item' is equal to the current node's data, then the item has been found and return true. \\
\textbf{Lines 61-64:}  If the string 'item' is less than the current node's data, then add 'L, ' to 'path' and continue the search by recursively calling lookup and passing in the left child of the current node. \\
\textbf{Lines 65-69:}  In any other senario, add 'R, ' to 'path' and continue the search by recursively calling lookup and passing in the right child of the current node. \\


%----------------------------------------------------------------------------------------
%   End BST
%----------------------------------------------------------------------------------------

\pagebreak

%----------------------------------------------------------------------------------------
%   Graphs
%----------------------------------------------------------------------------------------
\section{Graphs}
\subsection{Vertex Class}

% Insert C++ code with line numbers
\begin{linenumbers}
\begin{lstlisting}[language=C++, caption={Linear Search}, label={code:example}]
class Vertex {
public:
    int vertexID;
    bool processed;
    vector<Vertex*> neighbors;

    // Constructor
    Vertex(int id) : vertexID(id), processed(false) {}

    // Add a neighbor to the vertex
    void addNeighbor(Vertex* neighbor) {
        neighbors.push_back(neighbor);
    }    
};

\end{lstlisting}
\end{linenumbers}
\nolinenumbers

\textbf{Lines 70-77:} Define a C++ class called ’Vertex’ which contains three variables, ’int vertexID’, ’bool processed’, and ’vector<Vertex*> neighbors’. ’int vertexID’ stores the ID of the vertex. ’bool processed’ stores a boolean which tells the traversal functions whether or not the vertex has been processed. ’vector<Vertex*> neighbors’ stores pointers to neighboring vectors. \\
\textbf{Lines 80-82:} This function adds a vertex to the 'neighbors' vector.

\subsection{Graph Class}
\subsubsection{Deconstructor and Find Vertex Function}
% Insert C++ code with line numbers
\begin{linenumbers}
\begin{lstlisting}[language=C++, caption={Deconstructor and Find Vertex Function}, label={code:example}]
class Graph {
public:
    vector<Vertex> vertices;

    // Destructor to clear vertices
    ~Graph() {
        vertices.clear();
    }

    // Find a vertex by ID
    Vertex* findVertex(int id) {
        for (auto& vertex : vertices) {
            if (vertex.vertexID == id) {
                return &vertex;
            }
        }
        return nullptr;
    }
\end{lstlisting}
\end{linenumbers}
\nolinenumbers

\textbf{Line 86:}  Declare a vector which stores all the vertices in the graph. \\
\textbf{Lines 89-91:} This is a deconstructor which clears all the 'vertices' vertex, essentially erasing the graph. \\
\textbf{Lines 94-101:} This 'findVertex' function takes in a vertex ID and uses a ranged for loop that iterates through each vertex in the 'vertices' vector. If the vertexID of the current vertex equals the ID passed into the function, return a pointer to the current vertex. Return 'nullptr' if the ID is not found.



\subsubsection{Add Vertex and Add Edge Functions}
% Insert C++ code with line numbers
\begin{linenumbers}
\begin{lstlisting}[language=C++, caption={Add Vertex and Add Edge Functions}, label={code:example}]
// Add a new vertex to the graph
void addVertex(int id) {
    vertices.push_back(Vertex(id));
}

// Add an edge between two vertices
void addEdge(int id1, int id2) {
    Vertex* v1 = findVertex(id1);
    Vertex* v2 = findVertex(id2);

    if (v1 && v2) {
        v1->addNeighbor(v2);
        v2->addNeighbor(v1);
    }
}

\end{lstlisting}
\end{linenumbers}
\nolinenumbers

\textbf{Lines 103-105:} This 'addVertex' function takes in an id, creates a new Vertex object and adds it to the 'vertices' vector. \\
\textbf{Lines 108-116:} This 'addEdge' function takes in two vertex ID's, calls 'findVertex' to locate both of the vertices with the corresponding ID's. Then, it calls the 'addNeighbor' function for both of the vertices, adding each other as neighbors.


\subsubsection{Print Matrix Function}
% Insert C++ code with line numbers
\begin{linenumbers}
\begin{lstlisting}[language=C++, caption={Print Matrix Function}, label={code:example}]
// Print the graph as an adjacency matrix
void printMatrix() const {
    // Set the width for column headers
    int columnWidth = 3;

    // Print column headers
    cout << setw(columnWidth) << " ";
    for (const auto& vertex : vertices) {
        cout << setw(columnWidth) << vertex.vertexID;
    }
    cout << endl;

    // Print rows
    for (const auto& vertex : vertices) {
        cout << setw(columnWidth) << vertex.vertexID;
        for (const auto& otherVertex : vertices) {
            bool isNeighbor = false;
            for (const auto& neighbor : vertex.neighbors) {
                if (neighbor->vertexID == otherVertex.vertexID) {
                    isNeighbor = true;
                    break;
                }
            }
            cout << setw(columnWidth) << (isNeighbor ? "1" : "0");
        }
        cout << endl;
    }
}

\end{lstlisting}
\end{linenumbers}
\nolinenumbers

\textbf{Line 120:} Set the column width to ensure everything aligns well. \\
\textbf{Lines 123-127:} Use a ranged for loop to print the column headers. Iterate through each vertex in the 'vertices' vector and print the vertexID. Use 'setw' to set the column width for each vertex. \\
\textbf{Lines 130-144:} Use nested for loops to print the matrix body. Iterate through each vertex in the 'vertices' vector and print the vertex ID as the row header. Iterate through each 'vertex' in the 'vertices' vector again, and check if the current vertex is neighbors with the other vertex. It does this by iterating through the neighbors of the current vertex. If the other vertex is found in the list of neighbors, 'isNeighbor' is set to true. The function then prints "1" if the vertices are connected and "0" if they are not.

\subsubsection{Print Adjacency List Function}
% Insert C++ code with line numbers
\begin{linenumbers}
\begin{lstlisting}[language=C++, caption={Print Adjacency List Function}, label={code:example}]
    // Print the graph as an adjacency list
    void printAdjacencyList() const {
        for (const auto& vertex : vertices) {
            cout << "[" << vertex.vertexID << "] ";
            for (const auto& neighbor : vertex.neighbors) {
                cout << neighbor->vertexID << " ";
            }
            cout << endl;
        }
    }

\end{lstlisting}
\end{linenumbers}
\nolinenumbers

\textbf{Lines 146-148:} Loop through each vertex in the 'vertices' vector and print the 'vertexID'. \\
\textbf{Lines 149-154:} Loop through each vertex, printing the vertices in the 'neighbors' vector.


\subsubsection{Print as Linked Objects Function}
% Insert C++ code with line numbers
\begin{linenumbers}
\begin{lstlisting}[language=C++, caption={Print as Linked Objects Function}, label={code:example}]
    // Print the graph as linked objects
    void printLinkedObjects() const {
        for (const auto& vertex : vertices) {
            cout << "Vertex " << vertex.vertexID << ":" << endl;
            cout << "id\t\t " << vertex.vertexID << endl;
            cout << "processed\t " << boolalpha << vertex.processed << endl;
            cout << "neighbors\t [";
            for (size_t i = 0; i < vertex.neighbors.size(); ++i) {
                cout << vertex.neighbors[i]->vertexID;
                if (i < vertex.neighbors.size() - 1) {
                    cout << ",";
                }
            }
            cout << "]" << endl << endl;
        }
    }

\end{lstlisting}
\end{linenumbers}
\nolinenumbers

\textbf{Lines 156-161:} Loop through each vertex in the 'vertices' vector. Print the vertex number, the 'vertexID', and the 'processed' boolean. \\
\textbf{Lines 162-170:} Loop through the 'neighbors' vector of each vertex, printing each neighbor.

\subsubsection{Depth-First Search}
% Insert C++ code with line numbers
\begin{linenumbers}
\begin{lstlisting}[language=C++, caption={Depth-First Search}, label={code:example}]
    // Depth-first traversal function
    void DFS(Vertex* currentVertex, int& comparisons) {
        if (currentVertex && !currentVertex->processed) {
            cout << currentVertex->vertexID << " ";
            currentVertex->processed = true;

            for (Vertex* neighbor : currentVertex->neighbors) {
                comparisons++;
                if (!neighbor->processed) {
                    DFS(neighbor, comparisons);
                }
            }
        }
    }

    // Wrapper function for DFS to handle different starting points
    void performDFS() {
        for (auto& vertex : vertices) {
            vertex.processed = false;
        }

        int comparisons = 0;

        for (auto& vertex : vertices) {
            comparisons++;
            if (!vertex.processed) {
                DFS(&vertex, comparisons);
            }
        }

        cout << "\nNumber of comparisons: " << comparisons << endl;
    }

\end{lstlisting}
\end{linenumbers}
\nolinenumbers

\textbf{Lines 172-184:} If the current vertex is not 'nullptr' and not processed, print the current vertex, recursively call 'DFS' on every neighbor that has not been processed, and increment comparisons. \\
\textbf{Lines 187-190:} Loop through each vertex in the 'vertices' vector and set processed to 'false'. \\
\textbf{Line 192:} Initialize comparisons. \\
\textbf{Lines 194-199:} Loop through each vertex in the 'vertices' vector. Increment comparisons. If the vertex has not been processed, run the DFS function, passing in that vertex. \\
\textbf{Line 201:} Print the number of comparisons. \\

\subsubsection{Breadth-First Search}
% Insert C++ code with line numbers
\begin{linenumbers}
\begin{lstlisting}[language=C++, caption={Breadth-First Search}, label={code:example}]
    // Breadth-first traversal function
    void BFS(Vertex* startVertex, int& comparisons) const {
        if (!startVertex) {
            return;
        }

        queue<Vertex*> queue;
        queue.push(startVertex);

        startVertex->processed = true;

        while (!queue.empty()) {
            Vertex* currentVertex = queue.front();
            queue.pop();

            cout << currentVertex->vertexID << " ";

            for (Vertex* neighbor : currentVertex->neighbors) {
                comparisons++;
                if (!neighbor->processed) {
                    queue.push(neighbor);
                    neighbor->processed = true;
                }
            }
        }
    }

    // Wrapper function for BFS to handle different starting points
    void performBFS() {
        for (auto& vertex : vertices) {
            vertex.processed = false;
        }

        int comparisons = 0;

        for (auto& vertex : vertices) {
            comparisons++;
            if (!vertex.processed) {
                BFS(&vertex, comparisons);
            }
        }

        cout << "\nNumber of comparisons: " << comparisons << endl;
    }
};

\end{lstlisting}
\end{linenumbers}
\nolinenumbers

\textbf{Lines 204-207:} If the start vertex is 'nullptr', return. \\
\textbf{Lines 209-212:} Use a queue to keep track of the vertices. Push 'startVertex' to the queue and set processed to true. \\
\textbf{Lines 214-218:} While the queue is not empty, set 'currentVertex' as the vertex at the front of the queue, dequeue it, and print it. \\
\textbf{Lines 220-228:} Loop through each vertex in the current vertex's 'neighbors' vector. If the neighbor is not processed, push it to the queue, set processed to true, and increment comparisons. \\
\textbf{Lines 231-234:} Loop through each vertex in the 'vertices' vector and set processed to 'false'. \\
\textbf{Line 236:} Initialize comparisons. \\
\textbf{Lines 238-243:} Loop through each vertex in the 'vertices' vector. Increment comparisons. If the vertex has not been processed, run the BFS function, passing in that vertex. \\
\textbf{Line 245:} Print the number of comparisons. \\

%----------------------------------------------------------------------------------------
%   End Graphs
%----------------------------------------------------------------------------------------

\pagebreak

%----------------------------------------------------------------------------------------
%   Main Function
%----------------------------------------------------------------------------------------
\section{Main Function}
\subsection{Read 'magicitems.txt' and store items into BST}
% Insert C++ code with line numbers
\begin{linenumbers}
\begin{lstlisting}[language=C++, caption={Store Items in BST}, label={code:example}]
int main() {

    // Read all lines of magicitems.txt and put them in BST
    // Open the magicitems.txt file
    ifstream magicFile("magicitems.txt");

    // Handle failure
    if (!magicFile)
    {
        cerr << "Failed to open magicitems.txt" << endl;
        return 1;
    }

    Node* root = nullptr;

    int index = 0;
    string line;
    while (getline(magicFile, line)) {
        cout << "Inserting " << line << " Path: ";
        root = insert(root, line);
        cout << endl;
        index++;
    }

    magicFile.close();
\end{lstlisting}
\end{linenumbers}
\nolinenumbers

\textbf{Line 252:} Open the file 'magicitems.txt' for reading. \\
\textbf{Lines 255-259:} Check if the file was successfully opened. If the file cannot be opened, print an error message and exit the program. \\
\textbf{Line 261:} Initialize root node as 'nullptr'. \\
\textbf{Lines 263-270:} Loop through the file, inserting each line into the BST using the 'insert' function \\
\textbf{Line 272:} Close the file.


\subsection{In Order Traversal}
% Insert C++ code with line numbers
\begin{linenumbers}
\begin{lstlisting}[language=C++, caption={Merge Sort and Load Hash Table}, label={code:example}]
cout << endl << "BST In-Order Traversal: " << endl;
inOrderTraversal(root);
cout << endl;
\end{lstlisting}
\end{linenumbers}
\nolinenumbers

\textbf{Lines 273-275:} Call the 'inOrderTraversal' function passing in the root node.

\subsection{Conduct Lookup}
% Insert C++ code with line numbers
\begin{linenumbers}
\begin{lstlisting}[language=C++, caption={Conduct Searching}, label={code:example}]
// Read magicitems-find-in-bst.txt and look up each item in the BST
ifstream lookupFile("magicitems-find-in-bst.txt");

if (!lookupFile) {
    cerr << "Failed to open magicitems-find-in-bst.txt" << endl;
    return 1;
}

int totalComparisons = 0;
int totalLookups = 0;

while (getline(lookupFile, line)) {
    int comparisons = 0;
    string path = "";
    bool found = lookup(root, line, comparisons, path);
    totalLookups++;
    totalComparisons += comparisons;
    if (found) {
        cout << "Item found: " << line << endl << "Path: " << path << endl << "Comparisons: " << comparisons << endl << endl;
    } else {
        cout << "Item not found" << endl;
    }
}

lookupFile.close();
double averageComparisons = static_cast<double>(totalComparisons) / totalLookups;
printf("Average comparisons: %.2f\n", averageComparisons);
\end{lstlisting}
\end{linenumbers}
\nolinenumbers

\textbf{Line 277:} Open the file 'magicitems-find-in-bst.txt' for reading. \\
\textbf{Lines 279-282:} Check if the file was successfully opened. If the file cannot be opened, print an error message and exit the program. \\
\textbf{Lines 284-285:} Initialize 'totalComparisons' and 'totalLookups' \\
\textbf{Lines 287-290:} Loop through each line in the file. Initialize 'comparisons' to 0 and initialize the path. Initialize 'found' as the return value from the 'lookup' function which we call by passing in the root, line comparisons, and path. \\
\textbf{Lines 291:} Increment 'totalLookups' to keep track of the number of lookups made. \\
\textbf{Line 292:} Add 'comparisons' to 'totalComparisons'. \\
\textbf{Lines 293-298:} If the item was found, print the item, path, and number of comparisons. \\
\textbf{Line 300:} Close the file. \\
\textbf{Lines 301-102:} Convert 'totalComparisons' to a double and divide by 'totalLookups' to get the average. Print average comparisons to two decimal places. \\

\subsection{Read 'graphs1.txt' and Call The Graph Functions}
% Insert C++ code with line numbers
\begin{linenumbers}
\begin{lstlisting}[language=C++, caption={Read 'graphs1.txt' and Call The Graph Functions}, label={code:example}]
// Open the file
ifstream graphFile("graphs1.txt");

// Handle failure
if (!graphFile)
{
    cerr << "Failed to open graphs1.txt" << endl;
    return 1;
}

// Create a pointer to a Graph to hold the current graph
Graph* currentGraph = nullptr;

// Process each line in the file
while (getline(graphFile, line)) {
    istringstream iss(line);
    string command;
    iss >> command;

    if (command == "new" && iss >> command && command == "graph") {
        // "new graph" command
        if (currentGraph) {
            cout << "\nMatrix:\n";
            currentGraph->printMatrix();
            cout << "\nAdjacency List:\n";
            currentGraph->printAdjacencyList();
            cout << "\nLinked Objects:\n";
            currentGraph->printLinkedObjects();
            cout << "\nDepth-First Traversal:\n";
            currentGraph->performDFS();
            cout << "\nBreadth-First Traversal:\n";
            currentGraph->performBFS();
            delete currentGraph;
        }
        currentGraph = new Graph;
        cout << "\nCreated a new graph.\n";
    }
    else if (command == "add") {
        string subcommand;
        iss >> subcommand;

        if (subcommand == "vertex") {
            // "add vertex" command
            int id;
            if (iss >> id) {
                currentGraph->addVertex(id);
            }
        }
        else if (subcommand == "edge") {
            // "add edge" command
            int id1, id2;
            char hyphen;
            if (iss >> id1 >> hyphen >> id2 && hyphen == '-') {
                currentGraph->addEdge(id1, id2);
            }
        }
    }
}
\end{lstlisting}
\end{linenumbers}
\nolinenumbers

\textbf{Line 304:} Open the file 'graphs1.txt' for reading. \\
\textbf{Lines 307-311:} Check if the file was successfully opened. If the file cannot be opened, print an error message and exit the program. \\
\textbf{Line 314:} Initialize a pointer to a graph 'currentGraph' and set it to 'nullptr'. \\
\textbf{Line 305:} Loop through and read the file line by line. \\
\textbf{Lines 317-339:} If the line reads 'new graph', check if a graph already exists. If one does, then run all the functions for that graph, and then delete it. Finally, create a new graph. \\
\textbf{Lines 340-350:} If the line reads 'add vertex', run the 'addVertex' function, passing in the vertexID that is on the line. \\
\textbf{Lines 351-360:} If the line reads 'add add', run the 'addEdge' function, passing in the two vertexID's that are on the line. \\


\subsection{Print The Final Graph}
% Insert C++ code with line numbers
\begin{linenumbers}
\begin{lstlisting}[language=C++, caption={Read 'graphs1.txt' and Call The Graph Functions}, label={code:example}]
    // Print and delete the final graph
    if (currentGraph) {
        cout << "\Matrix:\n";
        currentGraph->printMatrix();
        cout << "\nAdjacency List:\n";
        currentGraph->printAdjacencyList();
        cout << "\nLinked Objects:\n";
        currentGraph->printLinkedObjects();
        cout << "\nDepth-First Traversal:\n";
        currentGraph->performDFS();
        cout << "\nBreadth-First Traversal:\n";
        currentGraph->performBFS();
        delete currentGraph;
    }

    // Close the file
    graphFile.close();

    return 0;
}

\end{lstlisting}
\end{linenumbers}
\nolinenumbers

\textbf{Lines 362-274:} Call all the functions for the final graph and delete it. \\
\textbf{Line 377:} Close the file. \\


%----------------------------------------------------------------------------------------
%   End Main Function
%----------------------------------------------------------------------------------------

\pagebreak

%----------------------------------------------------------------------------------------
%   Results
%----------------------------------------------------------------------------------------

\section{Analysis}

\subsection{BST Lookup}
The BST lookup function has a worst case time complexity of $O(n)$. This is because if the nodes are already in sorted order, we would visit every node until the target item is found; worst case being that the target item is the last item. However, we expect our tree to be relatively balanced. In the code, the function will recursively call itself until the target item is found. With each recursive call, we cut the tree in half, so we can expect the asymptotic running time to be $O(logn)$.

\subsection{Depth-First Search}
The graph's depth-first search function has a worst case time complexity of $O(n + e)$ where n is the number of vertices and e is the number of edges (number of neighbors). In the code, the 'preformDFS' function iterates over all vertices once, calling the DFS function, in $O(n)$ time. The DFS function iterates through all the neighbors of each vertex in $O(e)$ time, resulting in a time complexity of $O(n + e)$.

\subsection{Breadth-First Search}
The graph's bepth-first search function also has a worst case time complexity of $O(n + e)$ for the same reasons as depth-first search. In the code, the 'preformBFS' function iterates over all vertices once, calling the BFS function, in $O(n)$ time. The BFS function iterates through all the neighbors of each vertex in $O(e)$ time, resulting in a time complexity of $O(n + e)$.


%----------------------------------------------------------------------------------------
%   end Results
%----------------------------------------------------------------------------------------
\pagebreak


\end{document}
