%%%%%%%%%%%%%%%%%%%%%%%%%%%%%%%%%%%%%%%%%
%
% CMPT xxx
% Some Semester
% Lab/Assignment/Project X
%
%%%%%%%%%%%%%%%%%%%%%%%%%%%%%%%%%%%%%%%%%

%%%%%%%%%%%%%%%%%%%%%%%%%%%%%%%%%%%%%%%%%
% Short Sectioned Assignment
% LaTeX Template
% Version 1.0 (5/5/12)
%
% This template has been downloaded from: http://www.LaTeXTemplates.com
% Original author: % Frits Wenneker (http://www.howtotex.com)
% License: CC BY-NC-SA 3.0 (http://creativecommons.org/licenses/by-nc-sa/3.0/)
% Modified by Alan G. Labouseur  - alan@labouseur.com
%
%%%%%%%%%%%%%%%%%%%%%%%%%%%%%%%%%%%%%%%%%

%----------------------------------------------------------------------------------------
%	PACKAGES AND OTHER DOCUMENT CONFIGURATIONS
%----------------------------------------------------------------------------------------

\documentclass[letterpaper, 10pt,DIV=13]{scrartcl} 

\usepackage[T1]{fontenc} % Use 8-bit encoding that has 256 glyphs
\usepackage[english]{babel} % English language/hyphenation
\usepackage{amsmath,amsfonts,amsthm,xfrac} % Math packages
\usepackage{sectsty} % Allows customizing section commands
\usepackage{graphicx}
\usepackage[lined,linesnumbered,commentsnumbered]{algorithm2e}
\usepackage{listings}
\usepackage{parskip}
\usepackage{lastpage}
\usepackage{lineno}
\usepackage{float}

\allsectionsfont{\normalfont\scshape} % Make all section titles in default font and small caps.

\usepackage{fancyhdr} % Custom headers and footers
\pagestyle{fancyplain} % Makes all pages in the document conform to the custom headers and footers

\fancyhead{} % No page header - if you want one, create it in the same way as the footers below
\fancyfoot[L]{} % Empty left footer
\fancyfoot[C]{} % Empty center footer
\fancyfoot[R]{page \thepage\ of \pageref{LastPage}} % Page numbering for right footer

\renewcommand{\headrulewidth}{0pt} % Remove header underlines
\renewcommand{\footrulewidth}{0pt} % Remove footer underlines
\setlength{\headheight}{13.6pt} % Customize the height of the header

\numberwithin{equation}{section} % Number equations within sections (i.e. 1.1, 1.2, 2.1, 2.2 instead of 1, 2, 3, 4)
\numberwithin{figure}{section} % Number figures within sections (i.e. 1.1, 1.2, 2.1, 2.2 instead of 1, 2, 3, 4)
\numberwithin{table}{section} % Number tables within sections (i.e. 1.1, 1.2, 2.1, 2.2 instead of 1, 2, 3, 4)

\setlength\parindent{0pt} % Removes all indentation from paragraphs.

\binoppenalty=3000
\relpenalty=3000

%----------------------------------------------------------------------------------------
%	TITLE SECTION
%----------------------------------------------------------------------------------------

\newcommand{\horrule}[1]{\rule{\linewidth}{#1}} % Create horizontal rule command with 1 argument of height

\title{	
   \normalfont \normalsize 
   \textsc{CMPT 435 - Fall 2023 - Dr. Labouseur} \\[10pt] % Header stuff.
   \horrule{0.5pt} \\[0.25cm] 	% Top horizontal rule
   \huge Assignment One  \\     	    % Assignment title
   \horrule{0.5pt} \\[0.25cm] 	% Bottom horizontal rule
}

\author{Alex Tocci \\ \normalsize Mason.Tocci1@Marist.edu}

\date{\normalsize\today} 	% Today's date.

\begin{document}
\maketitle % Print the title

%----------------------------------------------------------------------------------------
%   start Stacks and Queues
%----------------------------------------------------------------------------------------
\section{Stack and Queue}

\subsection{Creating Node Class}
% Insert C++ code with line numbers
\begin{linenumbers}
\begin{lstlisting}[language=C++, caption={Node Class}, label={code:example}]
        #include <iostream>
        #include <fstream>
        #include <string>
        // Include for std::random_device
        #include <random>
        
        using namespace std;
        
        // Define a Node class for the stack and queue
        class Node {
        public:
            char data;
            Node* next;
        
            Node(char value) : data(value), next(nullptr) {}
        };
\end{lstlisting}
\end{linenumbers}
\nolinenumbers

\textbf{Lines 1-5:} First we must include the necessary C++ libraries for our program to run. \\
\textbf{Line 7:} This allows us to use standard C++ functions without the prefix 'std::'. \\
\textbf{Lines 10-13:} Define a C++ class called 'Node' which contains two variables, 'char data' and 'Node* next'. 'Char data' stores the data associated with the node. 'Node* next' points to the next node in the queue or stack or is set to 'nullptr', which indicates that there is no next node. \\
\textbf{Line 15:} This is a Node class constructor, which initializes the Node object


\subsection{Creating Stack Class}
% Insert C++ code with line numbers
\begin{linenumbers}
\begin{lstlisting}[language=C++, caption={Stack Class}, label={code:example}]
// Create a stack using the Node class
class Stack {
private:
    // Point to the top of the stack
    Node* top;

public:
    Stack() : top(nullptr) {}

    // Push a value onto the stack
    void push(char value) {
        Node* newNode = new Node(value);
        newNode -> next = top;
        top = newNode;
    }

    // Pop the top value from the stack
    char pop() {
        if (isEmpty()) {
            cerr << "Stack is empty." << endl;
            return -1;
        }
        char value = top->data;
        Node* temp = top;
        top = top -> next;
        delete temp;
        return value;
    }

    // Check if the stack is empty
    bool isEmpty() {
        return top == nullptr;
    }
};
\end{lstlisting}
\end{linenumbers}
\nolinenumbers

\textbf{Lines 18-21:} Define a class called 'Stack' that contains a 'Node* top' variable which keeps track of the top element in the stack. \\
\textbf{Line 24:} This is the 'Stack' class constructor which initializes the top pointer to 'nullptr', indicating an empty stack. \\
\textbf{Line 27:} This function is used to add an element onto the stack. \\
\textbf{Line 28:} Create a new Node with the given value. \\
\textbf{Line 29 :} Set the 'next' pointer of the new 'Node' to the current top of the stack. \\
\textbf{Line 30:} Update the 'top' pointer so it points to the new 'Node', making it the top of the stack. \\
\textbf{Line 34:} This function is used to remove the top element from the stack. \\
\textbf{Line 39:} If the stack is not empty, it gets the value stored in the current top Node. \\
\textbf{Line 40:} Create a temporary pointer to the current top Node. \\
\textbf{Line 41:} Update the 'top' pointer so it points to the next Node in the stack. \\
\textbf{Lines 42-43:} Delete the previous top Node and return the popped value. \\
\textbf{Lines 47- 49:} This function is used to check if the stack is empty. If the 'top' pointer is 'nullptr' it returns true. Meaning the stack is empty.

\subsection{Creating Queue Class}
% Insert C++ code with line numbers
\begin{linenumbers}
\begin{lstlisting}[language=C++, caption={Queue Class}, label={code:example}]
// Create a queue using the Node class
class Queue {
private:
    Node* front;
    Node* back;

public:
    Queue() : front(nullptr), back(nullptr) {}

    // Enqueue a value into the queue
    void enqueue(char value) {
        Node* newNode = new Node(value);
        if (isEmpty()) {
            front = back = newNode;
        }
        else {
            back -> next = newNode;
            back = newNode;
        }
    }

    // Dequeue a value from the queue
    char dequeue() {
        if (isEmpty()) {
            cerr << "Queue is empty." << endl;
            return -1;
        }
        char value = front->data;
        Node* temp = front;
        front = front -> next;
        delete temp;
        return value;
    }

    // Check if the queue is empty
    bool isEmpty() {
        return front == nullptr;
    }
};
\end{lstlisting}
\end{linenumbers}
\nolinenumbers

\textbf{Lines 52-55:} Define a class called 'Queue' that contains two varables, 'Node* front' and 'Node* back'. The front pointer keeps track of the front element in the queue. The back pointer keeps track of the back element in the queue. \\
\textbf{Line 58:} Initialize both 'front' and 'back' pointers to 'nullptr', an empty queue. \\
\textbf{Line 61:} This function is used to add an element to the queue. \\
\textbf{Line 62:} Create a new Node with the given value. \\ 
\textbf{Lines 63-65:} If the queue is empty, make both the 'front' and 'back' pointers point to the new 'Node'. This makes the new 'Node' both the front and back of the queue. \\
\textbf{Lines 66-69:} If the queue is not empty, the new Node is added to the back of the queue, and the 'back' pointer is updated to the new 'Node'. \\
\textbf{Line 73:} This function is used to remove the front element from the queue. \\
\textbf{Line 78:} If the queue is not empty, get the value stored in the current front Node. \\
\textbf{Line 79:} Create a temporary pointer to the current front Node. \\
\textbf{Line 80:} Update the 'front' pointer so it points to the next Node in the queue, removing the current front Node. \\
\textbf{Lines 81-83:} Delete the previous front Node and return the dequeued value. \\
\textbf{Lines 86-89:} This function is used to check if the queue is empty. If the 'front' pointer is 'nullptr', it returns true. Meaning the queue is empty.

\subsection{Check for palindrome}
% Insert C++ code with line numbers
\begin{linenumbers}
\begin{lstlisting}[language=C++, caption={Check for Palindrome}, label={code:example}]
// Check if a string is a palindrome (ignoring spaces and capitalization)
bool isPalindrome(const string& str) {
    Stack stack;
    Queue queue;

    // Push characters to the stack and queue (ignoring spaces and converting to lowercase)
    for (char i : str) {
        if (!isspace(i)) {
            stack.push(tolower(i));
            queue.enqueue(tolower(i));
        }
    }

    // Compare characters from the stack and queue
    while (!stack.isEmpty() && !queue.isEmpty()) {
        if (stack.pop() != queue.dequeue()) {
            // Not a palindrome
            return false;
        }
    }

    return true;
}
\end{lstlisting}
\end{linenumbers}
\nolinenumbers

\textbf{Lines 91-93:} Create a 'Stack' object and a 'Queue' object. \\
\textbf{Line 97:} Check if a character is not a space. \\
\textbf{Lines 98-100:} If 'i' is not a space, push the lowercase characters onto the 'stack' and enqueue the lowercase characters into the 'queue'. \\
\textbf{Line 104:} This 'while' loop continues as long as both the 'stack' and the 'queue' are not empty. \\
\textbf{Line 105-107:} Compare the characters at the top of the 'stack' and the front of the 'queue'. If the characters are not equal, it means that the string is not a palindrome, returning false.

\subsection{Shuffle}
% Insert C++ code with line numbers
\begin{linenumbers}
\begin{lstlisting}[language=C++, caption={Shuffle}, label={code:example}]
// Shuffle the array using the Knuth shuffle algorithm with std::random_device seeding
void shuffle(string* arr, int size) {
    random_device rd;
    srand(rd());

    for (int i = size - 1; i > 0; --i) {
        // Generate a random index between 0 and i
        int j = rand() % (i + 1); 
        // Swap elements at i and j
        swap(arr[i], arr[j]);     
    }
}
\end{lstlisting}
\end{linenumbers}
\nolinenumbers

\textbf{Lines 114-116:} Declare a random device named 'rd'. Random device is used to generate random numbers. Then, call 'srand(rd())' with 'rd' as the seed value to get the random number \\
\textbf{Lines 118-124:} This 'for' loop shuffles the elements of the array using the Knuth shuffle. We get a random array index and swap it with the current element in the loop.

%----------------------------------------------------------------------------------------
%   end PROBLEM ONE
%----------------------------------------------------------------------------------------

\pagebreak

%----------------------------------------------------------------------------------------
%   start PROBLEM TWO
%----------------------------------------------------------------------------------------

\section{Sorting}
\subsection{Selection Sort}
% Insert C++ code with line numbers
\begin{linenumbers}
\begin{lstlisting}[language=C++, caption={Selection Sort}, label={code:example}]
// Selection sort 
void selectionSort(string* arr, int size, int& comparisons) {
    for (int i = 0; i < size - 1; ++i) {
        int index = i;
        for (int j = i + 1; j < size; ++j) {
            // Convert both strings to lowercase for comparison
            for (char& ch : arr[j]) {
                ch = tolower(ch);
            }
            for (char& ch : arr[index]) {
                ch = tolower(ch);
            }
            comparisons++;
            // Compare the lowercase strings
            if (arr[j] < arr[index]) {
                index = j;
            }
        }
        // If a smaller element was found, swap it with the current element
        if (index != i) {
            swap(arr[i], arr[index]);
        }
    }
}
\end{lstlisting}
\end{linenumbers}
\nolinenumbers

\textbf{Lines 126-128:} This loop takes the minimum element in the unsorted part of the array and puts it in its correct position. \\
\textbf{Lines 129-136:} This for loop compares each element in the unsorted part of the array with the element at index 'i'. It converts the current element 'arr[index]' and the next element 'arr[j]' to lowercase. \\
\textbf{Line 137:} Increment 'comparisons' to keep track of the number of comparisons made. \\
\textbf{Lines 139-142:} Compare the lowercase strings. If the next element is less than the current element, update 'index' to 'j'. \\
\textbf{Lines 144-146:} After the inner 'for' loop completes, the 'index' variable contains the index of the smallest element in the unsorted part of the array. If 'index' is not equal to 'i', it means that a smaller element was found. If this is true, swap elements 'i' and 'index'.

\subsection{Insertion Sort}
% Insert C++ code with line numbers
\begin{linenumbers}
\begin{lstlisting}[language=C++, caption={Insertion Sort}, label={code:example}]
// Insertion sort
void insertionSort(string* arr, int size, int& comparisons) {
    for (int i = 1; i < size; ++i) {
        int j = i;
        while (j > 0) {
            // Convert both strings to lowercase for comparison
            for (char& ch : arr[j]) {
                ch = tolower(ch);
            }
            for (char& ch : arr[j - 1]) {
                ch = tolower(ch);
            }
            comparisons++;
            // Compare the lowercase strings
            if (arr[j] < arr[j - 1]) {
                swap(arr[j], arr[j - 1]);
                --j;
            }
            else {
                break;
            }
        }
    }
}
\end{lstlisting}
\end{linenumbers}
\nolinenumbers

\textbf{Lines 150-152:} This loop inserts each element into its correct position based on previously sorted elements. \\
\textbf{Lines 154-160:} Convert the elements at index 'j' and the index 'j - 1' to lowercase. \\
\textbf{Line 161:} Increment 'comparisons' to keep track of the number of comparisons made. \\
\textbf{Line 163-166:} Compare the lowercase strings. If 'arr[j]' is less than 'arr[j - 1]', swap the two elements and decrement 'j' to continue sorting. \\
\textbf{Lines 167-169:}If the comparison is false, break out of the inner loop. 

\subsection{Merge Sort}
\subsubsection{Merge Function}
% Insert C++ code with line numbers
\begin{linenumbers}
\begin{lstlisting}[language=C++, caption={Merge Function}, label={code:example}]
// Function to merge two sorted subarrays
void merge(string* arr, int left, int mid, int right, int& comparisons) {
    // Calculate the sizes of the subarrays
    int sizeLeft = mid - left + 1;
    int sizeRight = right - mid;

    // Create temporary arrays for the subarrays
    string* leftArray = new string[sizeLeft];
    string* rightArray = new string[sizeRight];

    // Copy data to temporary arrays
    for (int i = 0; i < sizeLeft; ++i) {
        leftArray[i] = arr[left + i];
    }
    for (int i = 0; i < sizeRight; ++i) {
        rightArray[i] = arr[mid + 1 + i];
    }

    // Merge the two subarrays back into array
    // Index for the left subarray
    int i = 0; 
    // Index for the right subarray
    int j = 0; 
    // Index for the merged array
    int k = left; 

    while (i < sizeLeft && j < sizeRight) {
        // Convert characters to lowercase for case-insensitive comparison
        for (char& ch : leftArray[i]) {
            ch = tolower(ch);
        }
        for (char& ch : rightArray[j]) {
            ch = tolower(ch);
        }

        comparisons++;
        // Compare and merge based on lowercase strings
        if (leftArray[i] <= rightArray[j]) {
            arr[k] = leftArray[i];
            ++i;
        }
        else {
            arr[k] = rightArray[j];
            ++j;
        }
        ++k;
    }

    // Copy the remaining elements of leftArray[], if any
    while (i < sizeLeft) {
        arr[k] = leftArray[i];
        ++i;
        ++k;
    }

    // Copy the remaining elements of rightArray[], if any
    while (j < sizeRight) {
        arr[k] = rightArray[j];
        ++j;
        ++k;
    }

    // Delete temporary arrays
    delete[] leftArray;
    delete[] rightArray;
}
\end{lstlisting}
\end{linenumbers}
\nolinenumbers

\textbf{Line 174:} This function, is a part of the merge sort algorithm and involves merging two sorted subarrays into a single sorted array. \\
\textbf{Line 176-177:} Calculate the sizes of the two subarrays. \\
\textbf{Lines 180-181:} Create temporary arrays for the left and right subarray. \\
\textbf{Lines 184-189:} Two 'for' loops copy the data from the original array into the subarrays.  \\
\textbf{Lines 193-197:} Merge the subarrays back into the original array. Initialize, three index variables: 'i' for the left subarray, 'j' for the right subarray, 'k' for the merged array.  \\
\textbf{Lines 201-206:} Elements in the left and right subarrays are converted to lowercase. \\
\textbf{Lines 201-206:}Increment 'comparisons' to keep track of the number of comparisons made. \\
\textbf{Lines 210-219:} Elements are compared based on their lowercase strings. The smaller element between the left subarray and right subarray is copied into the merged array. The index of the smaller element is incremented. Index 'k' is also incremented. \\
\textbf{Lines 222-233:} The two 'while' loops copy the remaining elements in the left and right subarrays to the merged array. \\
\textbf{Lines 236-238:} Delete the left and right subarrays.

\subsubsection{mergeSort Function}
% Insert C++ code with line numbers
\begin{linenumbers}
\begin{lstlisting}[language=C++, caption={mergeSort Function}, label={code:example}]
// Merge sort 
void mergeSort(string* arr, int left, int right, int& comparisons) {
    if (left < right) {
        // Find the middle point
        int mid = left + (right - left) / 2;

        // Recursively sort the first and second halves
        mergeSort(arr, left, mid, comparisons);
        mergeSort(arr, mid + 1, right, comparisons);

        // Merge the sorted halves
        merge(arr, left, mid, right, comparisons);
    }
}
\end{lstlisting}
\end{linenumbers}
\nolinenumbers

\textbf{Lines 240-243:}  If the left index of the current subarray is less than the right index of the current subarray, find the middle point. Calculate the midpoint by taking the average of 'left' and 'right'. The midpoint is the place where the array will be split. \\
\textbf{Lines 246-247:} The function recursively calls itself to sort the two halves of the current subarray. The first recursive call sorts the left half of the subarray and the second recursive call sorts the right half of the subarray. These recursive calls split the subarrays into smaller and smaller arrays until they contain only one element. \\
\textbf{Line 250:} Once the left and right halves of the subarray are sorted, the 'merge' function is called to merge them back together into a single sorted subarray.

\subsection{Quick Sort}
\subsubsection{Partition Function}
% Insert C++ code with line numbers
\begin{linenumbers}
\begin{lstlisting}[language=C++, caption={Partition Function}, label={code:example}]
// Partition function for quick sort
int partition(string* arr, int low, int high, int& comparisons) {
    // Choose the rightmost element as the pivot
    string pivot = arr[high]; 
    // Index of the smaller element
    int i = (low - 1); 

    for (int j = low; j <= high - 1; ++j) {
        // Convert both strings to lowercase for comparison
        for (char& ch : arr[j]) {
            ch = tolower(ch);
        }
        for (char& ch : pivot) {
            ch = tolower(ch);
        }

        comparisons++;
        // Compare the lowercase strings
        if (arr[j] < pivot) {
            ++i; 
            swap(arr[i], arr[j]);
        }
    }

    // Swap the pivot element
    swap(arr[i + 1], arr[high]);
    return (i + 1);
}
\end{lstlisting}
\end{linenumbers}
\nolinenumbers

\textbf{Lines 256-258:} Select the far right element in the given subarray as the pivot. 'i' is initialized to 'low - 1' and keeps track of the index of the smaller element during partitioning. \\
\textbf{Line 260:} This loop compares each element with the pivot and sorts them. \\
\textbf{Lines 262-267:} Inside the loop, both the current element and the pivot are converted to lowercase. \\
\textbf{Line 269:} Increment 'comparisons' to keep track of the number of comparisons made. \\
\textbf{Lines 271-274:} Compare the lowercase strings of the current element and the pivot. If the current element is less than the pivot, increment 'i' to mark the position of the smaller element and swap the element at index 'i' with the element at index 'j'. \\
\textbf{Lines 278-280:} Swap the pivot with the element at index 'i + 1' which places the pivot in its correct sorted position. Return the index '(i + 1)', the correct position of the pivot element in the sorted array.

\subsubsection{quickSort Function}
% Insert C++ code with line numbers
\begin{linenumbers}
\begin{lstlisting}[language=C++, caption={quickSort Function}, label={code:example}]
// Quick sort
void quickSort(string* arr, int low, int high, int& comparisons) {
    if (low < high) {
        // Find the pivot element such that
        // element smaller than pivot are on the left and
        // elements greater than pivot are on the right
        int pivotIndex = partition(arr, low, high, comparisons);

        // Recursively sort the subarrays
        quickSort(arr, low, pivotIndex - 1, comparisons);
        quickSort(arr, pivotIndex + 1, high, comparisons);
    }
}
\end{lstlisting}
\end{linenumbers}
\nolinenumbers

\textbf{Line 283-287:} If low is less than the high, find a pivot element and partition the subarray. \\
\textbf{Lines 290-293:} After partitioning the subarray, the 'quickSort' function is called recursively twice to sort the left and right subarrays. The first recursive call sorts the left subarray and the second recursive call sorts the right subarray. The recursive calls continue until all subarrays contain zero or one element.


%----------------------------------------------------------------------------------------
%   end PROBLEM Two
%----------------------------------------------------------------------------------------

\pagebreak

%----------------------------------------------------------------------------------------
%   start PROBLEM Three
%----------------------------------------------------------------------------------------
\section{Main Function}
\subsection{Test Stack and Queue}
% Insert C++ code with line numbers
\begin{linenumbers}
\begin{lstlisting}[language=C++, caption={Test Stack and Queue}, label={code:example}]
int main() {
    // Test the stack
    Stack testStack;
    testStack.push('A');
    testStack.push('B');
    testStack.push('C');

    cout << "Stack contents: ";
    while (!testStack.isEmpty()) {
        cout << testStack.pop() << " ";
    }
    cout << endl;

    // Test the queue
    Queue testQueue;
    testQueue.enqueue('A');
    testQueue.enqueue('B');
    testQueue.enqueue('C');

    cout << "Queue contents: ";
    while (!testQueue.isEmpty()) {
        cout << testQueue.dequeue() << " ";
    }
    cout << endl;
\end{lstlisting}
\end{linenumbers}
\nolinenumbers

\textbf{Lines 296-299:} Create an instance of a stack and push three characters onto it. The elements are added to the top of the stack in the order they are pushed. \\
\textbf{Lines 301-305:} The 'while' loop will continue as long as the stack is not empty. Within the loop, pop the top element from the stack and print it to the standard output. \\
\textbf{Lines 309-311:} Create an instance of a queue and enqueue three characters onto it. The elements are added to the back of the queue in the order they are enqueued. \\
\textbf{Lines 313-317:} The 'while' loop will continue as long as the queue is not empty. Within the loop, dequeue the front element from the queue and print it to the standard output.

\subsection{Read 'magicitems.txt' and store items into an array}
% Insert C++ code with line numbers
\begin{linenumbers}
\begin{lstlisting}[language=C++, caption={Store Items in Array}, label={code:example}]
// Read all lines of magicitems.txt and put them in an array
// Open the magicitems.txt file
ifstream file("magicitems.txt");

// Handle failure
if (!file) 
{
    cerr << "Failed to open magicitems.txt" << endl;
    return 1;
}

// Count the number of lines in the file
int magicItemsSize = 0;
string line;
while (getline(file, line)) 
{
    magicItemsSize++;
}

// Close and reopen the file to read from the beginning
file.close();
file.open("magicitems.txt");

// Create a dynamically allocated array
string* magicItemsArray = new string[magicItemsSize];

// Read the file line by line and store each line in the array
int index = 0;
while (getline(file, line)) 
{
    magicItemsArray[index] = line;
    index++;
}

// Close the file
file.close();
\end{lstlisting}
\end{linenumbers}
\nolinenumbers

\textbf{Line 320:} Open the file 'magicitems.txt' for reading. \\
\textbf{Lines 323-327:} Check if the file was successfully opened. If the file cannot be opened, print an error message and exit the program. \\
\textbf{Lines 330-335:} Read the file line by line and count the number of lines in the file. Increment the 'magicItemsSize' variable for each line read. \\
\textbf{Lines 338-339:} Close and reopen the file to read from the beginning. \\
\textbf{Line 342:} Create a dynamically allocated array of strings named 'magicItemsArray'. Set the size equal to 'magicItemsSize'. \\
\textbf{Lines 345-350:} Read the file line by line and store each line in the array. \\
\textbf{Line 353:} Close the file.

\subsection{Check for Palindromes}
% Insert C++ code with line numbers
\begin{linenumbers}
\begin{lstlisting}[language=C++, caption={Check for Palindromes}, label={code:example}]
cout << "\nPalindromes: " << endl;
// Check each line in the array for palindromes and print it if it is
for (int i = 0; i < magicItemsSize; ++i) {
    if (isPalindrome(magicItemsArray[i])) {
        cout << magicItemsArray[i] << endl;
    }
}
\end{lstlisting}
\end{linenumbers}
\nolinenumbers

\textbf{Lines 354-360:} The 'for' loop iterates over each line in the 'magicItemsArray'. Inside the loop, check if the current line in the array is a palindrome. If 'isPalindrome' returns true, it means that the current line is a palindrome and will be printed.

\subsection{Selection Sort}
% Insert C++ code with line numbers
\begin{linenumbers}
\begin{lstlisting}[language=C++, caption={Selection Sort}, label={code:example}]
// Shuffle before sorting
shuffle(magicItemsArray, magicItemsSize);

int comparisonsSelectionSort = 0;
// Sort magicItemsArray using selection sort
selectionSort(magicItemsArray, magicItemsSize, comparisonsSelectionSort);

cout << "\nSorted Magic Items (Selection Sort):" << endl;
for (int i = 0; i < magicItemsSize; ++i) {
    cout << magicItemsArray[i] << endl;
}
cout << "Comparisons in Selection Sort: " << comparisonsSelectionSort << endl;
\end{lstlisting}
\end{linenumbers}
\nolinenumbers

\textbf{Line 362:} Call the 'shuffle' function to shuffle the elements in 'magicItemsArray'. \\
\textbf{Lines 365-366:} Initialize an integer variable named 'comparisonsSelectionSort' to zero. Then, Sort the array using the selection sort algorithm. The 'comparisonsSelectionSort' variable is passed into the function, allowing the number of comparisons to be counted. \\
\textbf{Lines 368-371:} Use a 'for' loop to print each element in the sorted 'magicItemsArray'. \\
\textbf{Line 372:} Print the number of comparisons made during selection sort.

\subsection{Insertion Sort}
% Insert C++ code with line numbers
\begin{linenumbers}
\begin{lstlisting}[language=C++, caption={Insertion Sort}, label={code:example}]
// Shuffle before sorting again
shuffle(magicItemsArray, magicItemsSize);

int comparisonsInsertionSort = 0;
// Sort magicItemsArray using selection sort
insertionSort(magicItemsArray, magicItemsSize, comparisonsInsertionSort);

cout << "\nSorted Magic Items (Insertion Sort):" << endl;
for (int i = 0; i < magicItemsSize; ++i) {
    cout << magicItemsArray[i] << endl;
}
cout << "Comparisons in Insertion Sort: " << comparisonsInsertionSort << endl;
\end{lstlisting}
\end{linenumbers}
\nolinenumbers

\textbf{Line 374:} Shuffle the elements in 'magicItemsArray'. \\
\textbf{Lines 376-378:} Initialize an integer variable named 'comparisonsInsertionSort' to zero. Sort the array using the insertion sort algorithm. The 'comparisonsInsertionSort' variable is passed into the function, allowing the number of comparisons to be counted. \\
\textbf{Lines 380-383:} Use a 'for' loop to print each element in the sorted 'magicItemsArray'. \\
\textbf{Line 384:} Print the number of comparisons made during insertion sort.

\subsection{Merge Sort}
% Insert C++ code with line numbers
\begin{linenumbers}
\begin{lstlisting}[language=C++, caption={Merge Sort}, label={code:example}]
// Shuffle before sorting again
shuffle(magicItemsArray, magicItemsSize);

int comparisonsMergeSort = 0;
// Sort the strings using merge sort in a case-insensitive manner
mergeSort(magicItemsArray, 0, magicItemsSize - 1, comparisonsMergeSort);

cout << "\nSorted Magic Items (Merge Sort):" << endl;
for (int i = 0; i < magicItemsSize; ++i) {
    cout << magicItemsArray[i] << endl;
}
cout << "Comparisons in Merge Sort: " << comparisonsMergeSort << endl;
\end{lstlisting}
\end{linenumbers}
\nolinenumbers

\textbf{Line 386:} Shuffle the elements in 'magicItemsArray'. \\
\textbf{Lines 388-390:} Initialize an integer variable named 'comparisonsMergeSort' to zero. Sort the array using the merge sort algorithm. The 'comparisonsMergeSort' variable is passed into the function, allowing the number of comparisons to be counted. \\
\textbf{Lines 392-395:} Use a 'for' loop to print each element in the sorted 'magicItemsArray'. \\
\textbf{Line 396:} Print the number of comparisons made during merge sort.

\subsection{Quick Sort}
% Insert C++ code with line numbers
\begin{linenumbers}
\begin{lstlisting}[language=C++, caption={Quick Sort}, label={code:example}]
    // Shuffle before sorting again
    shuffle(magicItemsArray, magicItemsSize);

    // Sort magicItemsArray using quick sort
    int comparisonsQuickSort = 0; // Initialize comparisons counter
    quickSort(magicItemsArray, 0, magicItemsSize - 1, comparisonsQuickSort);

    cout << "\nSorted Magic Items (Quick Sort):" << endl;
    for (int i = 0; i < magicItemsSize; ++i) {
        cout << magicItemsArray[i] << endl;
    }

    cout << "Comparisons in Quick Sort: " << comparisonsQuickSort << endl;


    // Delete dynamically allocated memory
    delete[] magicItemsArray;

    return 0;
}
\end{lstlisting}
\end{linenumbers}
\nolinenumbers

\textbf{Line 398:} Shuffle the elements in 'magicItemsArray'. \\
\textbf{Lines 400-402:} Initialize an integer variable named 'comparisonsQuickSort' to zero. Sort the array using the quick sort algorithm. The 'comparisonsQuickSort' variable is passed into the function, allowing the number of comparisons to be counted. \\
\textbf{Lines 404-407:} Use a 'for' loop to print each element in the sorted 'magicItemsArray'. \\
\textbf{Line 409:} Print the number of comparisons made during quick sort. \\
\textbf{Line 413:} Delete dynamically allocated memory.

%----------------------------------------------------------------------------------------
%   end PROBLEM THREE
%----------------------------------------------------------------------------------------

\pagebreak

%----------------------------------------------------------------------------------------
%   start PROBLEM Four
%----------------------------------------------------------------------------------------

\section{Results}

\begin{table}[H]
\centering
\caption{Number of Comparisons}
\begin{tabular}{|c|c|c|c|c|c|c|}
\hline
 & 1 & 2 & 3 & 4 & 5 & Avg \\
\hline
Selection Sort & 221445 & 221445 & 221445 & 221445 & 221445 & 221445 \\
\hline
Insertion Sort & 111238 & 112774 & 107069 & 111565 & 109727 & 110474.6 \\
\hline
Merge Sort & 5435 & 5417 & 5440 & 5405 & 5423 & 5424 \\
\hline
Quick Sort & 7052 & 7407 & 6809 & 7234 & 6815 & 7063.4 \\
\hline
\end{tabular}
\end{table}

\subsection{Selection Sort}
Selection sort has a time complexity of $O(n^2)$ in the worst, average, and best cases. In this case, we always get 221,445 comparisons because the size of the array is fixed at 666, and selection sort performs the same number of comparisons regardless of the initial order of the array. In the code, the outer loop runs for '(size - 1)' iterations, and the inner loop runs for '(size)' iterations. This results in a total of $comparisons = (size - 1) * (size) / 2$ comparisons, plugging in 666 for size we get 221,445 comparisons.
\subsection{Insertion Sort} 
Insertion sort has a time complexity of $O(n^2)$ in its worst cases. In the code, the 'for' loop runs for '(size - 1)' iterations. The 'while' loop runs as long as the current element is less than its previous element; in the worst case, this would be the size of the array, 666. This results in a worst case of $comparisons = (size - 1) * (size) / 2$ comparisons,  plugging in 666 for size we get 221,445 comparisons. To get the average we would divide the worst case by 2, making our average 110,722.5.
\subsection{Merge Sort} 
Merge sort has a time complexity of $O(n log n)$ in its worst cases. In the code, the 'merge' function merges two sorted subarrays into a single sorted array. The time complexity of this function is $O(n)$. The 'mergeSort' function recursively divides the array into smaller subarrays until each subarray contains only one element. Then, it merges these subarrays back together in sorted order. The recursion takes $O(log n)$ time because the array is repeatedly halved. So we get a total time complexity is $O(n log n)$. Plugging in 666, our worst case is 6246.
\subsection{Quick Sort} 
Quick sort has a time complexity of $O(n log n)$ in its worst cases. In the code, the 'partitioning' function takes a pivot and rearranges the elements. The average time complexity of this function is $O(n)$. In the 'quicksort' function, each recursive call splits the array into two subarrays. The expected size of these subarrays is supposed to be nearly half. The recursion takes $O(log n)$ time because the array is repeatedly split. So we get an average time complexity of $O(n log n)$. Plugging in 666, our average case should be 6246.7.

%----------------------------------------------------------------------------------------
%   end PROBLEM Four
%----------------------------------------------------------------------------------------
\pagebreak


\end{document}
